\documentclass{article}
\usepackage[left=2cm,right=2cm,top=3cm,bottom=2cm]{geometry}
\usepackage{amsmath}
\usepackage{amssymb}

\title{Analysis and Approaches}
\author{Rohit Jethoe}
\date{February 4th 2025}

\begin{document}
\section{Sequences, Series, and Sigma Notation}
\subsection{Sequences}
A \textbf{sequence} = list written in a defined order
\begin{itemize}
        \item Each number making up a sequence is called a \textbf{term}
        \item Sequence is also called a \textbf{progression}
\end{itemize}
\hfill \break
Sequences may be \textbf{finite} or \textbf{infinite}
\begin{itemize}
        \item 7, 5, 3, 1, -1, -3 is a finite sequence.
        \item 7, 5, 3, 1, -1, -3, ... is a infinite sequence.
\end{itemize}
\hfill \break
A sequence can also be written in terms of \textbf{the general term} $ \{u_r\} $
$$ \{u_r\} = \{3r - 1\}, \text{where }r \in \mathbb{Z}^{+} \text{ represents the infinite sequence } 2,5,8,11, ...$$
$$ \{u_r\} = \{\dfrac{1}{r^2}\}, \text{where } r \in \mathbb{Z}^+, r \leq 5 \text{ respresents the finite sequence } 1,\dfrac{1}{4}, \dfrac{1}{9}, \dfrac{1}{16}, \dfrac{1}{25}$$
\hfill \break
\subsection{Series}
\end{document}